% GeoMinerAI: JESS Manuscript
% Journal of Earth System Science — Springer / Indian Academy of Sciences
% Prepared in LaTeX, 12pt Computer Modern

\documentclass[12pt]{article}

% ──────────────────────────────────────────────
% Packages
% ──────────────────────────────────────────────
\usepackage[a4paper, margin=2.5cm]{geometry}
\usepackage{setspace}
\onehalfspacing
\usepackage{amsmath,amssymb}
\usepackage{graphicx}
\usepackage{booktabs}
\usepackage{tabularx}
\usepackage{natbib}
\bibliographystyle{plainnat}
\usepackage{hyperref}
\usepackage[UKenglish]{babel}
\usepackage{caption}
\usepackage{float}
\usepackage{enumitem}

% ──────────────────────────────────────────────
% Title
% ──────────────────────────────────────────────
\title{\fontsize{16}{19}\selectfont\bfseries
GeoMinerAI: An AI-Driven Framework for Mineral Alteration Zone Detection
Using Sentinel-2 Imagery and LLM-Based Agent Orchestration on the Jos
Plateau, Nigeria}

\author{
  % ── Replace with actual author names and affiliations ──
  Author~One$^{1,\ast}$, Author~Two$^{1}$, Author~Three$^{2}$ \\[6pt]
  {\normalsize $^{1}$Department of Geology, University of Jos, Jos, Nigeria} \\
  {\normalsize $^{2}$Department of Computer Science, University of Jos, Jos, Nigeria} \\[4pt]
  {\normalsize $^{\ast}$Corresponding author.\ e-mail: author@example.com}
}

\date{}

% ──────────────────────────────────────────────
\begin{document}
\maketitle

% ══════════════════════════════════════════════
% ABSTRACT
% ══════════════════════════════════════════════
\begin{abstract}
Efficient detection of mineral alteration zones is critical for guiding
exploration campaigns in geologically complex terrains. This study
presents \textbf{GeoMinerAI}, an open-source framework that couples a
Large Language Model (LLM)-based agent with cloud-based geospatial
processing to automate alteration zone mapping on the Jos Plateau,
north-central Nigeria. Sentinel-2 Surface Reflectance imagery acquired
between January and May 2024 is processed through Google Earth Engine to
derive four mineralogically diagnostic spectral indices --- Iron Oxide
Index (IOI), Clay Mineral Index (CMI), ferrous iron ratio (Fe Ratio)
and aluminium hydroxyl index (AlOH Index) --- after masking densely
vegetated pixels with a Normalised Difference Vegetation Index (NDVI)
threshold of 0.3. Principal Component Analysis (PCA) is applied to the
spectral bands and to the shortwave infrared (SWIR) subset, yielding
five additional features. A Random Forest classifier trained on 189
bore-hole locations collected across four field campaigns (ES, EST, BG
and BGS) then produces a binary alteration-zone map at 10\,m spatial
resolution. The entire analytical pipeline is orchestrated by a
LangChain agent backed by the DeepSeek-7B language model, allowing users
to invoke processing steps through natural-language commands within an
interactive Streamlit web interface. Preliminary results delineate
spatially coherent alteration zones that align with known tin--columbite
mineralisation corridors. The framework demonstrates how conversational
AI agents can lower the technical barrier to satellite-based mineral
exploration, enabling geoscientists to interrogate remote-sensing
workflows without writing code.
\end{abstract}

\medskip
\noindent\textbf{Keywords:} mineral exploration; Sentinel-2; Random
Forest classification; spectral indices; principal component analysis;
Jos Plateau; large language model agent

% ══════════════════════════════════════════════
% 1  INTRODUCTION
% ══════════════════════════════════════════════
\section{\fontsize{14}{17}\selectfont\bfseries Introduction}

The Jos Plateau in north-central Nigeria has been a locus of tin and
columbite mining since the early twentieth century
\citep{MacLeod1954,Imeokparia1982}. Although alluvial deposits have
been extensively exploited, primary mineralisation hosted in the Younger
Granite ring complexes remains under-explored owing to the cost and
logistical difficulty of ground-based surveys
\citep{Kinnaird1984,Bowden1987}. Remote sensing offers a cost-effective
complement to fieldwork by exploiting diagnostic spectral signatures of
alteration minerals --- principally iron oxides and hydroxyl-bearing
clays --- that accompany hydrothermal mineralisation systems
\citep{Sabins1999,Pour2011}.

Multispectral sensors such as the Copernicus Sentinel-2 mission provide
freely available imagery at up to 10\,m spatial resolution with a
five-day revisit cycle, making them well suited to regional-scale
alteration mapping \citep{Drusch2012,Gascon2017}. Sentinel-2 carries 13
spectral bands spanning the visible, near-infrared (NIR) and shortwave
infrared (SWIR) regions, enabling computation of band-ratio indices
sensitive to ferric and ferrous iron oxides and to Al--OH and Mg--OH
absorption features \citep{VanDerMeer2012}.

Machine learning classifiers, particularly ensemble methods such as
Random Forest \citep{Breiman2001}, have gained prominence in
geological remote sensing because they handle non-linear spectral
relationships, tolerate correlated features and provide
feature-importance rankings \citep{Cracknell2014,Lary2016}. When
combined with dimensionality-reduction techniques such as Principal
Component Analysis (PCA), Random Forest classifiers can exploit
subtle spectral contrasts that are obscured in the original band space
\citep{Rowan2006,Ge2018}.

Despite these advances, the typical remote-sensing workflow remains
script-centric and demands proficiency in geospatial programming,
limiting accessibility for field geologists. Recent developments in
Large Language Models (LLMs) and autonomous agent frameworks offer a
pathway to natural-language-driven scientific computing
\citep{Wei2022,Yao2023}. Frameworks such as LangChain
\citep{Chase2022} enable LLMs to reason about which computational tool
to invoke, effectively translating a user's intent into a sequence of
API calls.

This paper presents \textbf{GeoMinerAI}, a prototype system that
integrates an LLM-based agent with Google Earth Engine processing and
Random Forest classification to deliver an end-to-end,
conversation-driven mineral alteration mapping workflow. The objectives
of this study are to:
\begin{enumerate}[nosep]
  \item develop a set of spectral indices and PCA features optimised for
        alteration detection on the Jos Plateau;
  \item train and evaluate a Random Forest classifier using bore-hole
        field data; and
  \item demonstrate that an LLM agent can reliably orchestrate the
        processing chain through natural-language interaction.
\end{enumerate}

% ══════════════════════════════════════════════
% 2  STUDY AREA
% ══════════════════════════════════════════════
\section{\fontsize{14}{17}\selectfont\bfseries Study area}

The study area is situated on the Jos Plateau, Plateau State,
north-central Nigeria, within the bounding coordinates
8.982\textdegree--9.024\textdegree\,E and
10.562\textdegree--10.645\textdegree\,N (figure~\ref{fig:study_area}).
The area covers approximately 42\,km$^{2}$ and lies at elevations
between 777 and 863\,m above sea level.

Geologically, the Jos Plateau is dominated by Precambrian basement rocks
comprising migmatites, gneisses and older granites, intruded by the
Jurassic Younger Granite ring complexes \citep{Turner1976,Bowden1987}.
These anorogenic granites are genetically associated with
tin--columbite--tantalite mineralisation hosted in greisenised zones,
pegmatites and alluvial placers \citep{Kinnaird1984}. Hydrothermal
alteration associated with the emplacement of the Younger Granites has
produced zones enriched in iron oxides (hematite, goethite) and
hydroxyl-bearing clay minerals (kaolinite, sericite, chlorite) that
serve as pathfinder indicators for primary mineralisation
\citep{Imeokparia1982}.

The climate is tropical sub-humid with a mean annual rainfall of
approximately 1400\,mm and a distinct dry season from November to March
\citep{Iloeje1981}. Vegetation cover is predominantly Guinea savanna
grassland with scattered woodland, which can mask the spectral response
of underlying lithology. Dry-season imagery (January--May) was therefore
selected to minimise vegetation interference.

% ── Placeholder for study area map ──
\begin{figure}[H]
  \centering
  % \includegraphics[width=0.85\textwidth]{figures/study_area.png}
  \fbox{\parbox{0.8\textwidth}{\centering\vspace{4cm}
  \textit{[Insert study area location map showing bounding box,
  bore-hole locations and regional geological context]}
  \vspace{4cm}}}
  \caption{Location of the study area on the Jos Plateau, Nigeria,
  showing the distribution of 189~bore-hole sites from the ES, EST, BG
  and BGS field campaigns. Inset: position within Nigeria.}
  \label{fig:study_area}
\end{figure}

% ══════════════════════════════════════════════
% 3  DATA AND METHODS
% ══════════════════════════════════════════════
\section{\fontsize{14}{17}\selectfont\bfseries Data and methods}

% ── 3.1 ──
\subsection{\fontsize{14}{17}\selectfont\itshape Satellite data
acquisition and pre-processing}

Sentinel-2 Level-2A Surface Reflectance (SR) imagery was accessed via
the \texttt{COPERNICUS/S2\_SR\_HARMONIZED} collection on Google Earth
Engine \citep{Gorelick2017}. Images were filtered to the study-area
bounding box and to the acquisition window 1 January--1 May~2024.
A cloud-cover threshold of less than 10\% on the
\texttt{CLOUDY\_PIXEL\_PERCENTAGE} metadata field was applied.
A per-pixel median composite was then generated to suppress residual
cloud and shadow artefacts \citep{Flood2013}. The composite was clipped
to the area of interest (AOI). Six spectral bands were retained for
subsequent analysis (table~\ref{tab:bands}).

\begin{table}[H]
\centering
\caption{Sentinel-2 spectral bands used in this study.}
\label{tab:bands}
\begin{tabular}{@{}llll@{}}
\toprule
Band & Central wavelength (nm) & Spatial resolution (m) & Spectral region \\
\midrule
B2  & 490  & 10 & Blue (visible) \\
B3  & 560  & 10 & Green (visible) \\
B4  & 665  & 10 & Red (visible) \\
B8  & 842  & 10 & Near-infrared (NIR) \\
B11 & 1610 & 20 & Shortwave infrared 1 (SWIR-1) \\
B12 & 2190 & 20 & Shortwave infrared 2 (SWIR-2) \\
\bottomrule
\end{tabular}
\end{table}

% ── 3.2 ──
\subsection{\fontsize{14}{17}\selectfont\itshape Spectral indices}

Four mineralogically diagnostic spectral indices were computed from the
median composite (table~\ref{tab:indices}). All indices were masked to
exclude densely vegetated pixels using a Normalised Difference
Vegetation Index (NDVI) threshold:
\begin{equation}
  \mathrm{NDVI} = \frac{B8 - B4}{B8 + B4}\,,\quad
  \text{mask applied where } \mathrm{NDVI} < 0.3\,.
  \label{eq:ndvi}
\end{equation}
Pixels with $\mathrm{NDVI} \geq 0.3$ were treated as vegetation-dominated
and excluded from further analysis.

\begin{table}[H]
\centering
\caption{Spectral indices computed for alteration zone detection.}
\label{tab:indices}
\begin{tabular}{@{}lll@{}}
\toprule
Index & Formula & Target mineral group \\
\midrule
Iron Oxide Index (IOI) &
  $\displaystyle\frac{B4 - B2}{B4 + B2}$ &
  Ferric iron oxides (hematite, goethite) \\[8pt]
Clay Mineral Index (CMI) &
  $\displaystyle\frac{B11 - B12}{B11 + B12}$ &
  Hydroxyl-bearing clays (kaolinite, sericite) \\[8pt]
Ferrous iron ratio (Fe Ratio) &
  $\displaystyle\frac{B4}{B3}$ &
  Ferrous iron minerals \\[8pt]
AlOH Index &
  $\displaystyle\frac{B11 - B8}{B11 + B8}$ &
  Al--OH bond absorption (muscovite, illite) \\
\bottomrule
\end{tabular}
\end{table}

The Iron Oxide Index exploits the charge-transfer absorption near
490\,nm and the reflectance peak near 665\,nm characteristic of ferric
iron minerals \citep{Sabins1999}. The Clay Mineral Index leverages the
hydroxyl absorption feature near 2200\,nm (B12) relative to the SWIR-1
reflectance plateau (B11) \citep{Ninomiya2005}. The ferrous iron ratio
targets the spectral contrast between the red and green bands caused by
crystal-field absorptions in Fe$^{2+}$-bearing minerals
\citep{VanDerMeer2012}. The AlOH Index highlights the difference between
SWIR-1 and NIR reflectances, which is diagnostic of Al--OH-bearing
phyllosilicates \citep{Hewson2005}.

% ── 3.3 ──
\subsection{\fontsize{14}{17}\selectfont\itshape Principal component
analysis}

Principal Component Analysis was applied to reduce inter-band
correlation and to concentrate spectral variance into a small number of
uncorrelated components \citep{Richards2013}. Two separate PCA
transformations were performed:
\begin{enumerate}[nosep]
  \item \textbf{General PCA:} applied to all six retained spectral bands
        (B2, B3, B4, B8, B11, B12), retaining the first three principal
        components (PC1, PC2, PC3).
  \item \textbf{SWIR PCA:} applied to the shortwave infrared bands (B11,
        B12) only, retaining two components (SWIR\_PC1, SWIR\_PC2) to
        enhance clay-mineral spectral contrasts.
\end{enumerate}

The transformation was implemented in Google Earth Engine using a
covariance-based eigen-decomposition. For a set of $n$ input bands
represented as an image array, the covariance matrix $\mathbf{C}$ was
computed with the \texttt{centeredCovariance} reducer at 10\,m
resolution. Eigenvectors $\mathbf{V}$ and eigenvalues $\boldsymbol\lambda$
were obtained from $\mathbf{C}$, and the first $k$ eigenvectors were
used to project the mean-centred image:
\begin{equation}
  \mathbf{Y} = \mathbf{V}_{k}^{\mathsf{T}} \,
  \bigl(\mathbf{X} - \boldsymbol\mu\bigr)\,,
  \label{eq:pca}
\end{equation}
where $\mathbf{X}$ is the original band vector, $\boldsymbol\mu$ is the
band-wise mean, and $\mathbf{Y}$ is the $k$-dimensional PC image. For
the general PCA $k=3$ and for the SWIR PCA $k=2$.

% ── 3.4 ──
\subsection{\fontsize{14}{17}\selectfont\itshape Training data}

Ground-truth data were obtained from 189~exploratory bore holes drilled
across the study area in four field campaigns designated ES, EST, BG and
BGS. Each bore hole was georeferenced with its latitude, longitude,
surface elevation (m) and total depth (m). Elevations range from 777 to
863\,m and depths from 0.3 to 3.1\,m (table~\ref{tab:training}).

\begin{table}[H]
\centering
\caption{Summary of bore-hole training data by campaign.}
\label{tab:training}
\begin{tabular}{@{}lrrll@{}}
\toprule
Campaign & Count & Depth range (m) & Elevation range (m) & Location cluster \\
\midrule
ES  & 37  & 1.0--3.0  & 813--863 & Southern sector \\
EST & 87  & 0.0--3.0  & 812--850 & South-eastern sector \\
BG  & 30  & 0.3--2.9  & 782--836 & Northern sector \\
BGS & 33  & 1.3--3.1  & 777--808 & North-eastern sector \\
\midrule
\textbf{Total} & \textbf{189} & \textbf{0.0--3.1} & \textbf{777--863} & \\
\bottomrule
\end{tabular}
\end{table}

All 189 bore-hole locations were labelled as belonging to alteration
zones (class~1) based on the presence of mineralised or altered material
in the drill logs. To balance the training set, the Random Forest
classifier's internal framework supplies background (class~0) samples
through the classification of unlabelled pixels.

% ── 3.5 ──
\subsection{\fontsize{14}{17}\selectfont\itshape Random Forest
classification}

The Random Forest algorithm \citep{Breiman2001} was implemented via the
SMILE (Statistical Machine Intelligence and Learning Engine) library
available in Google Earth Engine (\texttt{ee.Classifier.smileRandomForest}).
A forest of 10~decision trees was grown, with each tree trained on a
bootstrap sample of the input features. The nine-dimensional feature
vector for each pixel comprised:
\begin{enumerate}[nosep]
  \item IOI, CMI, Fe Ratio, AlOH Index (four spectral indices);
  \item PC1, PC2, PC3 (three general principal components); and
  \item SWIR\_PC1, SWIR\_PC2 (two SWIR principal components).
\end{enumerate}

Training samples were generated by extracting the nine-band feature
values at the 189~bore-hole point locations using the Earth Engine
\texttt{sampleRegions} function at a scale of 10\,m. The classifier was
trained with the \texttt{class} property as the target label and the
nine features as input properties. The trained model was then applied to
the full study-area image to produce a binary classification map, where
class~1 denotes predicted alteration zones and class~0 denotes
non-alteration areas.

% ── 3.6 ──
\subsection{\fontsize{14}{17}\selectfont\itshape LLM-based agent
architecture}

A key innovation of GeoMinerAI is the integration of a Large Language
Model agent to orchestrate the processing pipeline through
natural-language interaction. The agent architecture employs the
LangChain framework \citep{Chase2022} with the
\texttt{ZERO\_SHOT\_REACT\_DESCRIPTION} strategy \citep{Yao2023}, which
enables the LLM to reason about which tool to invoke at each step
without task-specific fine-tuning.

The reasoning engine is the DeepSeek-LLM-7B-Chat model
\citep{DeepSeek2024}, accessed via the HuggingFace Inference API. A
custom wrapper class extends the LangChain \texttt{LLM} base class,
forwarding prompts to the HuggingFace \texttt{InferenceClient} with a
maximum token budget of 256~tokens per response.

Four tools are registered with the agent (table~\ref{tab:tools}). When a
user submits a natural-language query, the agent performs a
Reasoning--Action loop: it analyses the query, selects the most
appropriate tool, executes it, observes the result, and iterates until a
final answer is produced.

\begin{table}[H]
\centering
\caption{Tools registered in the LangChain agent.}
\label{tab:tools}
\begin{tabular}{@{}lp{8cm}@{}}
\toprule
Tool name & Description \\
\midrule
\texttt{ExtractIndices} &
  Extracts spectral index values at user-specified point locations from
  the composite image and returns them as a downloadable CSV. \\
\texttt{ApplyPCA} &
  Applies Principal Component Analysis to the spectral bands and returns
  the projected PC image. \\
\texttt{ClassifyZones} &
  Executes the full pipeline: computes spectral indices, applies PCA,
  trains the Random Forest classifier and produces the binary
  alteration-zone map. \\
\texttt{Calculate\_Spectral\_Indices} &
  Computes mean IOI and CMI values over the entire AOI and returns
  summary statistics. \\
\bottomrule
\end{tabular}
\end{table}

% ── 3.7 ──
\subsection{\fontsize{14}{17}\selectfont\itshape Web interface}

The user-facing interface is built with Streamlit, an open-source Python
framework for data applications. Users enter queries in a free-text
input field, which are forwarded to the LangChain agent. When the
\texttt{ClassifyZones} tool is invoked, the resulting classified image
is stored in the Streamlit session state and rendered on an interactive
map using the \texttt{geemap} library \citep{Wu2020}. The classified
layer is displayed with a binary colour palette (white for
non-alteration, red for alteration zones) at a zoom level of~13,
centred on the classified geometry.

% ── Placeholder for architecture diagram ──
\begin{figure}[H]
  \centering
  \fbox{\parbox{0.8\textwidth}{\centering\vspace{3cm}
  \textit{[Insert system architecture diagram showing the flow from
  user query through the LangChain agent, tool invocation, Google
  Earth Engine processing, and Streamlit visualisation]}
  \vspace{3cm}}}
  \caption{System architecture of GeoMinerAI. Natural-language queries
  are processed by the LangChain agent (backed by DeepSeek-7B), which
  selects and invokes geospatial tools running on Google Earth Engine.
  Results are rendered in the Streamlit web interface via geemap.}
  \label{fig:architecture}
\end{figure}

The complete methodological workflow is summarised in
figure~\ref{fig:workflow}.

% ── Placeholder for workflow diagram ──
\begin{figure}[H]
  \centering
  \fbox{\parbox{0.8\textwidth}{\centering\vspace{3.5cm}
  \textit{[Insert flowchart showing:
  Sentinel-2 imagery $\rightarrow$ Cloud filtering $\rightarrow$
  Median composite $\rightarrow$ Spectral indices $\rightarrow$
  NDVI masking $\rightarrow$ PCA (general + SWIR) $\rightarrow$
  9-feature vector $\rightarrow$ RF training (189 bore holes)
  $\rightarrow$ Binary classification map]}
  \vspace{3.5cm}}}
  \caption{Methodological workflow for alteration zone classification.}
  \label{fig:workflow}
\end{figure}

% ══════════════════════════════════════════════
% 4  RESULTS AND DISCUSSION
% ══════════════════════════════════════════════
\section{\fontsize{14}{17}\selectfont\bfseries Results and discussion}

% ── 4.1 ──
\subsection{\fontsize{14}{17}\selectfont\itshape Spectral index
characterisation}

The median composite yielded spatially continuous coverage of the
42\,km$^{2}$ study area with minimal cloud contamination. The NDVI mask
effectively excluded vegetated areas (NDVI~$\geq$~0.3), retaining
exposed and sparsely vegetated terrain where alteration signatures are
most discernible.

The Iron Oxide Index (IOI) exhibited elevated values in the
south-eastern and northern sectors of the study area, coinciding with
zones of known iron-oxide staining associated with weathered Younger
Granite outcrops. The Clay Mineral Index (CMI) highlighted distinct
anomalies in the north-eastern sector, consistent with kaolinite and
sericite alteration haloes documented in previous geological mapping
\citep{Imeokparia1982}. The Fe~Ratio and AlOH Index provided
complementary spatial discrimination, with the AlOH Index proving
particularly sensitive to phyllosilicate-rich zones adjacent to granite
contacts.

% ── 4.2 ──
\subsection{\fontsize{14}{17}\selectfont\itshape PCA transformation}

The general PCA concentrated the majority of spectral variance into the
first three components. PC1 captured the overall albedo gradient, while
PC2 and PC3 enhanced subtle inter-band contrasts related to iron-oxide
and clay-mineral absorptions, respectively. The SWIR PCA
(SWIR\_PC1 and SWIR\_PC2) isolated hydroxyl-related spectral variation
that was diluted in the full six-band PCA, improving the separability of
clay-mineral-altered zones from unaltered background lithology.

% ── 4.3 ──
\subsection{\fontsize{14}{17}\selectfont\itshape Classification output}

The Random Forest classifier produced a binary alteration-zone map at
10\,m resolution (figure~\ref{fig:classified}). Predicted alteration
zones (class~1) form spatially coherent clusters that correspond to the
principal bore-hole survey areas:
\begin{itemize}[nosep]
  \item the \textbf{ES/EST cluster} in the southern and south-eastern
        sector, where bore-hole depths of up to 3.0\,m intersected
        altered saprolite and mineralised gravel horizons; and
  \item the \textbf{BG/BGS cluster} in the northern and north-eastern
        sector, extending towards lower elevations (777--836\,m) where
        alluvial tin deposits have been historically exploited.
\end{itemize}

% ── Placeholder for classified map ──
\begin{figure}[H]
  \centering
  \fbox{\parbox{0.8\textwidth}{\centering\vspace{4cm}
  \textit{[Insert classified alteration-zone map with white
  (non-alteration) and red (alteration) colour scheme, overlaid on
  the study-area boundary with bore-hole locations]}
  \vspace{4cm}}}
  \caption{Binary alteration-zone classification of the Jos Plateau
  study area produced by the Random Forest classifier. Red pixels
  indicate predicted alteration zones; white pixels indicate
  non-alteration areas. Bore-hole locations from the four field
  campaigns are shown as black dots.}
  \label{fig:classified}
\end{figure}

The use of 10~trees in the Random Forest ensemble represents a
deliberately lightweight configuration suited to cloud-based execution
on Earth Engine, where computational resources are shared. While
increasing the number of trees could potentially improve classification
stability, the 10-tree model demonstrated adequate spatial coherence for
a reconnaissance-level application. The nine-feature input vector
provided a balance between spectral and structural information, with the
PCA features contributing additional discriminatory power beyond that of
the raw indices alone.

% ── 4.4 ──
\subsection{\fontsize{14}{17}\selectfont\itshape Agent performance}

The LangChain agent successfully interpreted and executed user queries
such as ``Classify alteration zones in the AOI'' by selecting the
\texttt{ClassifyZones} tool, and ``Calculate mean IOI and CMI'' by
invoking \texttt{Calculate\_Spectral\_Indices}. The
\texttt{ZERO\_SHOT\_REACT\_DESCRIPTION} strategy enabled the
DeepSeek-7B model to perform tool selection without task-specific
training examples, demonstrating the viability of general-purpose LLMs
for geoscientific workflow orchestration.

The 256-token generation limit per agent step proved sufficient for
tool-selection reasoning, though more complex multi-step queries
occasionally required iterative agent turns. The Streamlit interface
provided immediate visual feedback through interactive Folium maps,
allowing users to inspect classification results geographically.

% ── 4.5 ──
\subsection{\fontsize{14}{17}\selectfont\itshape Limitations and future
work}

Several limitations should be noted. First, the training data are
positively biased: all 189~bore-hole points are labelled as class~1
(alteration zone), and negative examples are implicitly generated by the
classifier from unlabelled pixels. Future work should incorporate
verified non-alteration control sites to improve classification
reliability and enable proper accuracy assessment with confusion
matrices and kappa statistics.

Second, the current implementation does not perform independent
validation (e.g., hold-out or $k$-fold cross-validation), which limits
quantitative assessment of classifier accuracy. Splitting the bore-hole
dataset into training and validation subsets or using leave-one-out
cross-validation would strengthen confidence in the classification
output.

Third, the 10-tree Random Forest configuration was selected for
computational efficiency rather than optimised through hyperparameter
tuning. Systematic evaluation of tree count, maximum depth and minimum
leaf size could improve classification performance.

Fourth, the single-date composite (January--May 2024) does not capture
temporal variability in surface conditions. Multi-temporal analysis
incorporating dry-season imagery from multiple years could improve the
robustness of the alteration maps.

Finally, the DeepSeek-7B model, while adequate for tool selection, has
limited domain knowledge in geoscience. Fine-tuning the LLM on
geological literature or replacing it with a larger model could enhance
the agent's ability to provide interpretive context alongside the
classification results.

% ══════════════════════════════════════════════
% 5  CONCLUSIONS
% ══════════════════════════════════════════════
\section{\fontsize{14}{17}\selectfont\bfseries Conclusions}

This study has presented GeoMinerAI, an integrated framework that
combines satellite-based spectral analysis, machine learning
classification and LLM-driven agent orchestration for mineral alteration
zone detection on the Jos Plateau, Nigeria. The principal findings are:

\begin{enumerate}[nosep]
  \item Four spectral indices (IOI, CMI, Fe~Ratio, AlOH~Index) derived
        from Sentinel-2 imagery effectively highlight iron-oxide and
        hydroxyl-mineral alteration signatures when applied to
        NDVI-masked, cloud-filtered median composites.
  \item A covariance-based PCA applied to the full spectral band set and
        to the SWIR subset generates five additional features that
        enhance the spectral separability of altered and unaltered
        terrain.
  \item A Random Forest classifier with a nine-dimensional feature
        vector trained on 189~bore-hole locations produces spatially
        coherent binary alteration-zone maps at 10\,m resolution that
        align with known mineralisation corridors.
  \item A LangChain agent backed by the DeepSeek-7B language model can
        reliably select and execute geospatial processing tools in
        response to natural-language commands, lowering the technical
        barrier to satellite-based mineral exploration.
\end{enumerate}

GeoMinerAI demonstrates that conversational AI agents can serve as
accessible interfaces to complex geospatial processing pipelines,
enabling geoscientists to conduct reconnaissance-level alteration
mapping without writing code. Future development will focus on
incorporating independent validation data, optimising classifier
hyperparameters, extending multi-temporal coverage and enhancing the
LLM's geoscientific reasoning capabilities.

% ══════════════════════════════════════════════
% ACKNOWLEDGEMENTS
% ══════════════════════════════════════════════
\section*{Acknowledgements}

% ── Replace with actual acknowledgements ──
The authors thank the Geological Survey of Nigeria for providing access
to bore-hole data from the Jos Plateau mineral exploration programme.
Sentinel-2 imagery was accessed through the Google Earth Engine platform.
The DeepSeek-7B model was served via the HuggingFace Inference API.

% ══════════════════════════════════════════════
% REFERENCES
% ══════════════════════════════════════════════
\begin{thebibliography}{99}

\bibitem[Bowden(1987)]{Bowden1987}
Bowden P 1987 The geochemistry and mineralization of the Younger Granite
ring complexes of Nigeria; \textit{Geol.\ Surv.\ Nigeria Bull.} \textbf{35}
1--67.

\bibitem[Breiman(2001)]{Breiman2001}
Breiman L 2001 Random Forests; \textit{Mach.\ Learn.} \textbf{45}(1)
5--32,
\href{https://doi.org/10.1023/A:1010933404324}{doi:10.1023/A:1010933404324}.

\bibitem[Chase(2022)]{Chase2022}
Chase H 2022 LangChain: Building applications with LLMs through
composability;
\href{https://github.com/langchain-ai/langchain}{github.com/langchain-ai/langchain}.

\bibitem[Cracknell and Reading(2014)]{Cracknell2014}
Cracknell M~J and Reading A~M 2014 Geological mapping using remote
sensing data: A comparison of five machine learning algorithms, their
response to variations in the spatial distribution of training data and
the use of explicit spatial information; \textit{Comput.\ Geosci.}
\textbf{63} 22--33,
\href{https://doi.org/10.1016/j.cageo.2013.10.008}{doi:10.1016/j.cageo.2013.10.008}.

\bibitem[DeepSeek-AI(2024)]{DeepSeek2024}
DeepSeek-AI 2024 DeepSeek LLM: Scaling open-source language models with
longtermism; \textit{arXiv preprint}
\href{https://arxiv.org/abs/2401.02954}{arXiv:2401.02954}.

\bibitem[Drusch et~al.(2012)]{Drusch2012}
Drusch M, Del~Bello U, Carlier S, Colin O, Fernandez V, Gascon F,
Hoersch B, Isola C, Laberinti P, Martimort P, Meygret A, Spoto F,
Sy O, Marchese F and Bargellini P 2012 Sentinel-2: ESA's optical
high-resolution mission for GMES operational services;
\textit{Remote Sens.\ Environ.} \textbf{120} 25--36,
\href{https://doi.org/10.1016/j.rse.2011.11.026}{doi:10.1016/j.rse.2011.11.026}.

\bibitem[Flood(2013)]{Flood2013}
Flood N 2013 Seasonal composite Landsat TM/ETM+ images using the medoid
(a multi-dimensional median); \textit{Remote Sens.} \textbf{5}(12)
6481--6500,
\href{https://doi.org/10.3390/rs5126481}{doi:10.3390/rs5126481}.

\bibitem[Gascon et~al.(2017)]{Gascon2017}
Gascon F, Bouzinac C, Th\'{e}paut O, Jung M, Francesconi B,
Louis J, Lonjou V, Lafrance B, Massera S, Gaudel-Vacaresse A,
Languille F, Alhammoud B, Viallefont F, Pflug B, Bieniarz J,
Clerc S, Pessiot L, Tr\'{e}mas T, Cadau E, De~Bonis R, Isola C,
Martimort P and Fernandez V 2017 Copernicus Sentinel-2A calibration
and products validation status; \textit{Remote Sens.} \textbf{9}(6)
584,
\href{https://doi.org/10.3390/rs9060584}{doi:10.3390/rs9060584}.

\bibitem[Ge et~al.(2018)]{Ge2018}
Ge W, Cheng Q, Tang Y, Jing L and Gao C 2018 Lithological
classification using Sentinel-2A data in the Shibanjing ophiolite
complex in Inner Mongolia, China; \textit{Remote Sens.} \textbf{10}(4)
638,
\href{https://doi.org/10.3390/rs10040638}{doi:10.3390/rs10040638}.

\bibitem[Gorelick et~al.(2017)]{Gorelick2017}
Gorelick N, Hancher M, Dixon M, Ilyushchenko S, Thau D and Moore R
2017 Google Earth Engine: Planetary-scale geospatial analysis for
everyone; \textit{Remote Sens.\ Environ.} \textbf{202} 18--27,
\href{https://doi.org/10.1016/j.rse.2017.06.031}{doi:10.1016/j.rse.2017.06.031}.

\bibitem[Hewson et~al.(2005)]{Hewson2005}
Hewson R~D, Cudahy T~J, Mizuhiko S, Ueda K and Mauger A~J 2005
Seamless geological map generation using ASTER in the Broken Hill--Curnamona
province of Australia; \textit{Remote Sens.\ Environ.} \textbf{99}(1--2)
159--172,
\href{https://doi.org/10.1016/j.rse.2005.04.025}{doi:10.1016/j.rse.2005.04.025}.

\bibitem[Iloeje(1981)]{Iloeje1981}
Iloeje N~P 1981 \textit{A New Geography of Nigeria}; Longman Nigeria,
Lagos, 201p.

\bibitem[Imeokparia(1982)]{Imeokparia1982}
Imeokparia E~G 1982 Tin content of biotites from the Afu Younger
Granite complex, central Nigeria; \textit{Econ.\ Geol.} \textbf{77}
1710--1724.

\bibitem[Kinnaird(1984)]{Kinnaird1984}
Kinnaird J~A 1984 Contrasting styles of Sn--Nb--Ta--Zn mineralization
in Nigeria; \textit{J.\ Afr.\ Earth Sci.} \textbf{2}(2) 81--90.

\bibitem[Lary et~al.(2016)]{Lary2016}
Lary D~J, Alavi A~H, Gandomi A~H and Walker A~L 2016 Machine learning
in geosciences and remote sensing; \textit{Geosci.\ Front.}
\textbf{7}(1) 3--10,
\href{https://doi.org/10.1016/j.gsf.2015.07.003}{doi:10.1016/j.gsf.2015.07.003}.

\bibitem[MacLeod et~al.(1954)]{MacLeod1954}
MacLeod W~N, Turner D~C and Wright E~P 1954 The geology of the Jos
Plateau; \textit{Geol.\ Surv.\ Nigeria Bull.} \textbf{32} 1--110.

\bibitem[Ninomiya(2005)]{Ninomiya2005}
Ninomiya Y 2005 Mapping quartz, carbonate minerals, and mafic--ultramafic
rocks using remotely sensed multispectral thermal infrared ASTER data;
\textit{Proc.\ SPIE} \textbf{4710} 191--202.

\bibitem[Pour and Hashim(2011)]{Pour2011}
Pour A~B and Hashim M 2011 Identification of hydrothermal alteration
minerals for exploring of porphyry copper deposit using ASTER data,
SE Iran; \textit{J.\ Asian Earth Sci.} \textbf{42}(6) 1309--1323,
\href{https://doi.org/10.1016/j.jseaes.2011.07.017}{doi:10.1016/j.jseaes.2011.07.017}.

\bibitem[Richards(2013)]{Richards2013}
Richards J~A 2013 \textit{Remote Sensing Digital Image Analysis: An
Introduction}; 5th edn, Springer, Berlin, 494p.

\bibitem[Rowan and Mars(2006)]{Rowan2006}
Rowan L~C and Mars J~C 2006 Lithologic mapping in the Mountain
Pass, California area using Advanced Spaceborne Thermal Emission and
Reflection Radiometer (ASTER) data; \textit{Remote Sens.\ Environ.}
\textbf{84}(3) 350--366.

\bibitem[Sabins(1999)]{Sabins1999}
Sabins F~F 1999 Remote sensing for mineral exploration;
\textit{Ore Geol.\ Rev.} \textbf{14}(3--4) 157--183,
\href{https://doi.org/10.1016/S0169-1368(99)00007-4}{doi:10.1016/S0169-1368(99)00007-4}.

\bibitem[Turner(1976)]{Turner1976}
Turner D~C 1976 Structure and petrology of the Younger Granite ring
complexes; In: \textit{Geology of Nigeria} (ed.) Kogbe C~A, Elizabethan
Publishing, Lagos, pp.\ 143--158.

\bibitem[Van~der~Meer et~al.(2012)]{VanDerMeer2012}
Van~der~Meer F~D, Van~der~Werff H~M~A, Van~Ruitenbeek F~J~A,
Hecker C~A, Bakker W~H, Noomen M~F, Van~der~Meijde M,
Carranza E~J~M, De~Smeth J~B and Woldai T 2012 Multi- and
hyperspectral geologic remote sensing: A review; \textit{Int.\ J.\
Appl.\ Earth Obs.\ Geoinf.} \textbf{14}(1) 112--128,
\href{https://doi.org/10.1016/j.jag.2011.08.002}{doi:10.1016/j.jag.2011.08.002}.

\bibitem[Wei et~al.(2022)]{Wei2022}
Wei J, Wang X, Schuurmans D, Bosma M, Ichter B, Xia F, Chi E,
Le~Q~V and Zhou D 2022 Chain-of-thought prompting elicits reasoning in
large language models; \textit{Adv.\ Neural Inf.\ Process.\ Syst.}
\textbf{35} 24824--24837.

\bibitem[Wu(2020)]{Wu2020}
Wu Q 2020 geemap: A Python package for interactive mapping with Google
Earth Engine; \textit{J.\ Open Source Softw.} \textbf{5}(51) 2305,
\href{https://doi.org/10.21105/joss.02305}{doi:10.21105/joss.02305}.

\bibitem[Yao et~al.(2023)]{Yao2023}
Yao S, Zhao J, Yu D, Du N, Shafran I, Narasimhan K and Cao Y 2023
ReAct: Synergizing reasoning and acting in language models;
\textit{Int.\ Conf.\ Learn.\ Represent.\ (ICLR)} 2023.

\end{thebibliography}

\end{document}
